\documentclass{beamer}
\usepackage[utf8]{inputenc}
\usepackage{mathptmx}
\usepackage{microtype}
\usepackage{booktabs}
\usepackage[nice]{nicefrac}
\usepackage{amsmath}
\usepackage{amssymb}
\usepackage{graphicx}
\usepackage{fancyhdr}
\usepackage{hyperref}
\usepackage[german]{babel}
\usepackage{ulem}
\usepackage{tikz}

\newcommand\blfootnote[1]{%
  \begingroup
  \renewcommand\thefootnote{}\footnote{#1}%
  \addtocounter{footnote}{-1}%
  \endgroup
}
\newcommand{\citeS}[1]{\textsuperscript{\cite{#1}}}

%\input{includes/Ergebnisimplementierung}

\begin{document}
\section{Title page}
\title{\textbf {Quantum Computing Project\\[0,75cm]
Presentation\\[1,5cm]}}
\author{Hunter, Gennaro, Michael, Christoph}
\date{24th March 2015}
\maketitle

\section{Contents}

\begin{frame}<1>[label=Contents]
	\begin{enumerate}
		\item Contents

	\end{enumerate}
\end{frame}

%------------------------------------------------------------
%------------------------------------------------------------
%------------------------------------------------------------
%------------------------------------------------------------
%Einleitung
%------------------------------------------------------------
%------------------------------------------------------------
%------------------------------------------------------------
%------------------------------------------------------------
\section{Erster Teil}
%
%------------------------------------------------------------
%Versuchsüberblick
%------------------------------------------------------------
\againframe<11>{Inhaltsverzeichnis}

%------------------------------------------------------------
%Aufgabenstellung und Ziel
%------------------------------------------------------------
\begin{frame}<1>[label=Aufgabenstellung]
	\frametitle{Aufgabenstellung}
	%
	Aus den Daten sind folgende Größen zu bestimmen: 
	\begin{enumerate}
		\item Gesamte, leptonische und hadronische Breite und M$_\text{Z}$
		\item<2-> Die Vorwärts-Rückwärts-Asymmetrie beim Resonanzmaximum und der Weinbergwinkel
		\item<3> Die Anzahl der Neutrinogenerationen
	\end{enumerate}
\end{frame}
\againframe<2>{Aufgabenstellung}
\againframe<3>{Aufgabenstellung}


%------------------------------------------------------------
%------------------------------------------------------------
%------------------------------------------------------------
%Physikalische Grundlagen
%------------------------------------------------------------
%------------------------------------------------------------
%------------------------------------------------------------

%\againframe<2>{Inhaltsverzeichnis}


%------------------------------------------------------------
% Kurzer Wiederholung aus der Teilchenphysik
%------------------------------------------------------------
\againframe<21>{Inhaltsverzeichnis}

\begin{frame}
	\frametitle{Wiederholung der Teilchenphysik}
	\framesubtitle{Teilchen und fundamentale Wechselwirkugnen}
	%
	\begin{figure}
		\centering
		\includegraphics[width=0.3\textwidth]{Bilder/GrundlagenFermionen.jpg}
%		\caption{Familien der Fermionen \citeS{Fermionen}}
	\end{figure}
\end{frame}
\begin{frame}
	\frametitle{Wiederholung der Teilchenphysik}
	\framesubtitle{Teilchen und fundamentale Wechselwirkugnen}
	%
	\begin{figure}
	\centering
		\begin{tabular}{|l||l|l|}
		\hline
			Elektromagnetisch & Photon & Elektrische Ladung \\\hline
			Stark & Gluon & Farbe \\\hline
			Schwach & W$^\pm$, Z$^0$ & Schwache Ladung \\\hline
			Gravitation & hyp. Graviton & Masse \\\hline
		\end{tabular}
	\end{figure}
\end{frame}

\begin{frame}<1>[label=WQResonanz]
	\frametitle{Wiederholung der Teilchenphysik}
	\framesubtitle{Wirkungsquerschnitt und Resonanzkurven}
	%
	Wirkungsquerschnitt $\sigma$ und Luminosität $ L $
	\begin{align*}
		\dfrac{dn}{dt}=& \sigma L\\
		n =& \sigma \int L dt
	\end{align*}
	\visible<2->{
	Zerfallsbreite $ \Gamma $ und Lebensdauer $\tau$
	\begin{align*}
		\Gamma \cdot \tau = \hbar
	\end{align*}}
	\visible<3>{
	partielle Breite $\Gamma_{i}$ und Verzweigungsverhältnis $ BR $
	\begin{align*}
		BR_{i}= \dfrac{\Gamma_{i}}{\sum_{j}^{} \Gamma_{j}}
	\end{align*}}
	%	\begin{textblock}{8}(10.5,1)
	%			\includegraphics[width=4cm]{Bilder/Resonanzkurve.jpg}
	%	\end{textblock}
\end{frame}

\end{document}




%\begin{frame}
%\color[rgb]{0,0,0}
%\color<2-3>[rgb]{1,0,0}
%
%Dieser Text ist rot auf
%Folien 98 und 99 und auf allen anderen schwarz.
%
%\color[rgb]{0,0,0}
%\alt<2>{Diese Zeile ändert sich nur auf Folie 103.}
%{Dies ist nicht die 103. Folie.}
%
%\end{frame}