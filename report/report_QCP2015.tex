\documentclass[bibliography=totocnumbered]{article}
%\usepackage[T1]{fontenc} Verpixelt!

%\bibliographystyle{plain}

\usepackage{geometry}
\geometry{a4paper, left=40mm, right=40mm, top=45mm, bottom=35mm}

\usepackage[utf8]{inputenc}
%\usepackage[german]{babel}
%\usepackage{sistyle}
%\usepackage{paralist}
\usepackage{graphicx}
\usepackage{amsmath}
\usepackage{amssymb}
\usepackage{float}
\usepackage[usenames,dvipsnames]{color}
\usepackage{forloop}
\usepackage{icomma}
\usepackage{rotating}
\usepackage{multirow}
\usepackage{wasysym}
\usepackage{amsthm}

\usepackage[final]{pdfpages}

%\usepackage[style=authortitle-icomp]{biblatex}
%\usepackage[babel,german=guillemets]{csquotes}
%\bibliography{lit} 

%\usepackage{floatflt}



\usepackage[margin=5pt, font=small,labelfont=bf]{caption} %Sehr geiles Package
\usepackage[margin=10pt, list=true, font=small, labelfont=bf, labelformat=brace, position=top]{subcaption} %Sehr geiles Package


\usepackage{verbatim}
%\usepackage{sverb}
\usepackage{listings}

\definecolor{mygreen}{rgb}{0,0.6,0}
\definecolor{mygray}{rgb}{0.5,0.5,0.5}
\definecolor{mymauve}{rgb}{0.58,0,0.82}

\definecolor{mylinkColor}{rgb}{0,0.14,0.4}

\setlength{\fboxrule}{0.3mm}

\lstset{ %
  backgroundcolor=\color{white},   % choose the background color; you must add \usepackage{color} or \usepackage{xcolor}
  basicstyle=\tiny,        % the size of the fonts that are used for the code
  breakatwhitespace=false,         % sets if automatic breaks should only happen at whitespace
  breaklines=true,                 % sets automatic line breaking
  captionpos=b,                    % sets the caption-position to bottom
  commentstyle=\color{mygreen},    % comment style
  deletekeywords={...},            % if you want to delete keywords from the given language
  escapeinside={\%*}{*)},          % if you want to add LaTeX within your code
  extendedchars=true,              % lets you use non-ASCII characters; for 8-bits encodings only, does not work with UTF-8
  frame=single,                    % adds a frame around the code
  keepspaces=true,                 % keeps spaces in text, useful for keeping indentation of code (possibly needs columns=flexible)
  keywordstyle=\color{blue},       % keyword style
  language=Octave,                 % the language of the code
  morekeywords={*,...},            % if you want to add more keywords to the set
  numbers=left,                    % where to put the line-numbers; possible values are (none, left, right)
  numbersep=5pt,                   % how far the line-numbers are from the code
  numberstyle=\tiny\color{mygray}, % the style that is used for the line-numbers
  rulecolor=\color{black},         % if not set, the frame-color may be changed on line-breaks within not-black text (e.g. comments (green here))
  showspaces=false,                % show spaces everywhere adding particular underscores; it overrides 'showstringspaces'
  showstringspaces=false,          % underline spaces within strings only
  showtabs=false,                  % show tabs within strings adding particular underscores
  stepnumber=2,                    % the step between two line-numbers. If it's 1, each line will be numbered
  stringstyle=\color{mymauve},     % string literal style
  tabsize=2,                       % sets default tabsize to 2 spaces
  title=\lstname                   % show the filename of files included with \lstinputlisting; also try caption instead of title
}


%Kopfzeile
\usepackage{fancyhdr}
\usepackage{lastpage}
\pagestyle{fancy}
\fancyhf{} %Löscht Voreinstellungen
\lhead{\nouppercase{\leftmark}}
\rhead{Page \thepage \ of \pageref{LastPage}}
%Kopfzeile Ende

%-----------------------------------------------------------------------------------------
% TITELBLATT
%-----------------------------------------------------------------------------------------
\newcommand{\titlename}{}

\newcommand{\authorname}{Gennaro, Michael, Hunter, Christoph}

\newcommand{\citeS}[1]{\textsuperscript{\cite{#1}}}
\newcommand{\imgref}[1]{Fig.\,\ref{#1}}

%\newcommand{\captionS}[1]{\textbf{\caption{\normalfont{\small #1}}}}





{\small }

\title{
 \titlename\\
  \vspace{2cm}
  %\includegraphics[width=\textwidth]{Bilder/Title.PNG}
  \vspace{1cm}
}
\author{\authorname}
\date{}

\usepackage{hyperref} 
\usepackage{wrapfig}

\usepackage[all]{hypcap} %Damit Verlinkungen auf Bilder richtig gemacht werden




\hypersetup{
    bookmarks=true,         % show bookmarks bar?
    unicode=false,          % non-Latin characters in Acrobat’s bookmarks
    pdftoolbar=false,        % show Acrobat’s toolbar?
    pdfmenubar=true,        % show Acrobat’s menu?
    pdffitwindow=false,     % window fit to page when opened
    pdfstartview={FitH},    % fits the width of the page to the window
    pdftitle={\titlename},    % title
    pdfauthor={\authorname},     % author
    pdfsubject={},   % subject of the document
    pdfcreator={},   % creator of the document
    pdfproducer={}, % producer of the document
    pdfkeywords={}, % list of keywords
    pdfnewwindow=true,      % links in new window
    colorlinks=true,       % false: boxed links; true: colored links
    linkcolor=mylinkColor,          % color of internal links (change box color with linkbordercolor)
    citecolor=OliveGreen,        % color of links to bibliography
    filecolor=magenta,      % color of file links
    urlcolor=cyan           % color of external links
}

\newtheorem{definition}{Definition}[section]
\newtheorem{bsp}[definition]{Example}

\newtheoremstyle{NoticeStyle}% name of the style to be used
  {}% measure of space to leave above the theorem. E.g.: 3pt
  {}% measure of space to leave below the theorem. E.g.: 3pt
  {}% name of font to use in the body of the theorem
  {}% measure of space to indent
  {}% name of head font
  {}% punctuation between head and body
  {}% space after theorem head; " " = normal interword space
  {}% Manually specify head
 
\theoremstyle{NoticeStyle}

\newcommand{\Anmerkung}[1]{
	
	\vspace{3pt}
	\begin{tabular}{||p{0.9\textwidth}}
		\textsc{Notice:} {\small #1}
	\end{tabular}
	
	\vspace{3pt}
}


\begin{document}

\newpage
\begin{titlepage}
	\centering
	{\LARGE \textsc{The University of Edinburgh}}\\[5pt]
	{\large \textsc{School of Physics and Astronomy}}\\
	\vspace{80pt}
	
	\rule{\linewidth}{1pt}
	{
	\textbf{\LARGE Quantum Computing Project:\\Report}
	}
	\rule{\linewidth}{1pt}
	
%	\vspace{80pt}
%	{\large
%	A report\\as part of\\
%	\textbf{}\\
%	}
	\vspace{80pt}
	{\large
		
	}
	
	\vspace{50pt}
	{\large
	by\\
	\textsc{Gennaro, Michael, Hunter, Christoph}
	}
	\vfill
	{\large Edinburgh, 23rd of February 2015}
\end{titlepage}

\newpage


%-----------------------------------------------------------------------------------------
% INHALTSVERZEICHNIS
%-----------------------------------------------------------------------------------------
\tableofcontents
\newpage


%-----------------------------------------------------------------------------------------
% INTRODUCTION
%-----------------------------------------------------------------------------------------
%
\section{Introduction}


%-----------------------------------------------------------------------------------------
% THEORY
%-----------------------------------------------------------------------------------------
%
\section{Theory}

% - Explain why certain problems are rendered tractable by quantum computation with reference to the relevant concepts in quantum theory.

\subsection{Quantum Computer}

% - State some of the practical challenges in building a quantum computer.

\subsection{Grover's Algorithm}


%-----------------------------------------------------------------------------------------
% METHODS
%-----------------------------------------------------------------------------------------
%
\section{Methods}

% - Describing our medium-scale application program and the work of our a co-operative team
% - Describing collaborative development tools (Java) (such as version control systems (Git)).


%-----------------------------------------------------------------------------------------
% RESULTS
%-----------------------------------------------------------------------------------------
%
\section{Results}
% - Documentation that is comprehensible to a group of different programmers.


\subsection{Grover's Algorithm}
%Plot of Results


%-----------------------------------------------------------------------------------------
% DISCUSSION
%-----------------------------------------------------------------------------------------
%
\section{Discussion}

%-----------------------------------------------------------------------------------------
% CONCLUSION
%-----------------------------------------------------------------------------------------
%
\section{Conclusion}


%-----------------------------------------------------------------------------------------
% APPENDIX
%-----------------------------------------------------------------------------------------
%
\section{Appendix}
%

\newpage

\phantomsection
\addcontentsline{toc}{section}{References}

\phantomsection
\bibliographystyle{plaindin}
\bibliography{lit}


\end{document}